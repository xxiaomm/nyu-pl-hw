
\documentclass[11pt]{article}
\textheight=9.0in
\textwidth=6.5in
\topmargin=-.3in
\oddsidemargin=-0.2in
\itemsep=6pt

\newcommand{\ra}{\rightarrow}
\newcommand{\lam}{\lambda}
\newcommand{\comment}[1]{}

%IMPORTANT: Need to have comment.sty in the working directory.
\usepackage{comment}


\begin{document}


\begin{center}
{\bf
Programming Languages\\

CSCI-GA.2110.001 Fall 2021 \\
\vspace{2ex} 
Homework 1\\
Due Sunday, October 24 at 11:55pm}
\end{center}

\vspace{2ex}
\noindent
You should write the answers using Word, latex, etc., and upload them as a PDF document.  No implementation is required. Since there are drawings, you can take photos or scan the drawings and upload them separately.

\vspace{1ex}
\begin{enumerate} 
\item 
Provide regular expressions for defining the syntax of the following. You can only use concatenation, $|$, $*$, $\epsilon$ (the empty string), parentheses, and expressions of the form [A-Z], [0-3], [A-Z0-3], etc., to create regular expressions. That is, you cannot use $+$ or any kind of count indicator.

\begin{enumerate} 
\item 
Strings consisting of any number of lower case letters, upper case letters, and digits, such that the occurrences of upper case letters and digits are alternating.  That is, there cannot be two upper case letters in the string without a digit somewhere in between them and there cannot be two digits in the string without an upper case letter somewhere between them.  For example, ``aAbc3dEf4ghIjk5m'' would be an example of such a string. However, ``aAbc3dEfGhi4jk'' would not be a valid string, since there is no digit between E and G.


\item 
Positive and negative floating point literals that specify a positive or negative exponent, such as the following:  243.876E11 (representing $243.867 \times 10^{11}$) and -135.24E-4 (representing $-135.24 \times 10^{-4}$). There must be at least one digit before the decimal point and one digit after the decimal point (before the ``E'').

\item 
  Variable names that (1) must start with a letter, (2) may contain any number of letters, digits, and \_ (underscores), and (3) may not contain two consecutive underscores.

\end{enumerate} 

\item 
\begin{enumerate} 
\item Provide a simple context-free grammar for the language in which the following program is written. You can assume that the syntax of names and numbers are already defined using regular expressions (i.e. you don't have to define the syntax for names and numbers). In this language, a program consists of a sequence of function definitions.
\begin{verbatim} 
fun fac 0 = 1
 |  fac n = n * fac(n-1)

fun foo x y = 
   let val z = x + y 
       fun bar n = n * 2
   in bar z
   end
\end{verbatim} 

You only have to create grammar rules that are sufficient to parse the above program. 
Your starting non-terminal should be called PROG (for ``program'') and the above
program should be able to be derived from PROG.

As a hint to get you started, here are the first few productions in the CFG that I wrote:
\\
{\tt
PROG $\ra$ FUNS \\
FUNS $\ra$ FUN | FUN FUNS \\
FUN $\ra$ fun DEFS \\
DEFS $\ra$ DEF | DEF "|" DEFS \\
}

Feel free to use the above lines, but you certainly don't have to.

\item 
Draw the parse tree for the above program, based on a derivation using your grammar. The parse tree will be quite large, so you should separate it across several pages (a subtree on each page).
\end{enumerate} 

\item 
\begin{enumerate} 
\item 
Define the terms {\em static scoping} and {\em dynamic scoping}.

\item 
Give a simple example, in any language you like (actual or imaginary),
that would illustrate the difference between static and dynamic
scoping. That is, write a short piece of code whose result would be
different depending on whether static or dynamic scoping was used.

\item
  Why do all modern programming languages use static scoping? That is, what is
  the advantage of static scoping over dynamic scoping?
\end{enumerate} 


\item 
\begin{enumerate} 
\item
Draw the call stack for the following program that would exist when
the print statement is executed. Assume the language is statically
scoped.  Show at least the local variables and parameters (including
any closure), the static and dynamic links, and identify which stack
frame is for which procedure.
\begin{verbatim} 
procedure A()
   x: integer := 17;

   procedure B(procedure D)
      x: integer := 20;

      procedure C()
        procedure F(y:integer)
        begin (* F *)
           D(y)
        end; (* F *)
      begin (* C *)
         F(x);
      end; (* C *)

   begin (* B *)
      C();
   end (* B *)

   procedure  E(z:integer)
   begin (* E *)
      print(z, x);
   end; (* E *)

begin (* A *)
  B(E);
end; (* A *)
\end{verbatim} 
 
\item What would the program print?

\item Suppose, instead, that the language was dynamically scoped. What would
the program print?
\end{enumerate} 

\item 
For each of these parameter passing mechanisms, 
\begin{enumerate} 
\item pass by value
\item pass by reference
\item pass by value-result (assume the address of an argument is computed only once)
\item pass by name
\end{enumerate} 
state what the following program (in some Pascal-like language) would
print if that parameter passing mechanism was used:
\begin{quote}\begin{verbatim} 
program foo;
  var i,j: integer;
      a: array[1..10] of integer;

  procedure f(x,y:integer)
  begin
    x := x * 4;
    i := i + 1;
    y := a[i] * 5;
  end

begin
  for j := 1 to 10 do a[j] = j;
  i := 2;
  f(i,a[i]);
  for j := 1 to 10 do print(a[j]);
end.
\end{verbatim}\end{quote}


\item 

  In Ada, define a procedure containing two tasks such that (1) the first task prints the
  odd numbers from 1 to 99 in order, (2) the second task prints the even numbers
  from 2 to 100 in order, and (3) for each odd number N, the value of N and N+1 are
  printed concurrently but there is no other concurrent printing of numbers.  For example,
  1 and 2 are printed concurrently, and 3 and 4 are printed concurrently, but 1 and 2 must
  be printed before 3 and 4.

\end{enumerate} 
\end{document}
